\documentclass[paper=a4, fontsize=9pt]{article}
\usepackage[utf8]{inputenc}

\usepackage[a4paper]{geometry}
\geometry{top=2cm, bottom=2cm, left=2cm, right=2cm}
\setlength{\parskip}{2mm}

\usepackage{lipsum}

\usepackage[spanish]{babel}										
\usepackage[protrusion=true,expansion=true]{microtype}		    % Better typography
\usepackage{amsmath,amsfonts,amsthm}					                % Math packages
\usepackage[pdftex]{graphicx}									% Enable pdflatex
\usepackage[svgnames]{xcolor}									% Enabling colors by their 'svgnames'
\usepackage[hang, small, labelfont=bf,up,textfont=it,up]{caption}	% Custom captions under/above floats
\usepackage{epstopdf}											  	% Converts .eps to .pdf
\usepackage{subfig}												  	% Subfigures
\usepackage{booktabs}											  	% Nicer tables
\usepackage{fix-cm}												  	% Custom fontsizes

%opening
\title{Métodos estadísticos aplicados al baloncesto}
\author{Paula Moreno Blazquez}

\usepackage{Sweave}
\begin{document}
\Sconcordance{concordance:TFG_MetodosEstadisticosBaloncesto.tex:TFG_MetodosEstadisticosBaloncesto.Rnw:%
1 31 1 1 0 226 1 1 34 16 0 1 51 37 0 1 208 2 0 1 31 1 2 44 1 1 2 1 0 18 %
1 1 3 1 0 1 5 3 0 1 3 1 0 1 3 19 0 1 3 1 0 1 5 3 0 2 2 1 0 2 2 1 0 2 2 %
1 0 2 2 1 0 2 2 1 0 2 2 1 0 2 2 1 0 2 2 1 0 2 2 1 0 2 2 1 1 1 2 41 0 1 %
2 2 1 1 3 2 0 2 1 1 7 5 0 1 2 1 7 5 0 1 2 1 1 1 3 4 0 1 2 10 1 1 7 6 0 %
1 5 6 0 1 2 4 1}


\maketitle

\clearpage

\begin{abstract}

Hoy en día, el deporte es un hobby muy popular por todo el mundo. Des de pequeños, los niños practican algún tipo de deporte, especialmente aquellos que son de equipo. Eso nos lleva a querer saber más del deporte, más detalles, más información. Nos entra la curiosidad de "¿quién es el mejor jugador?", "Qué equipo es mejor?", o incluso intentar prevenir qué equipo ganará según sus resultados anteriores. Y gracias a los avances tecnológicos e informáticos, cada vez se nos facilita más poder seguir un deporte des de casa, ver la estadística de los deportistas e incluso hay plataformas o juegos que nos permiten ser, de manera virtual, managers de los clubs y, por lo tanto, nos facilitan mucha información que antes era más difícil de saber.

Eso hace que, de manera progresiva, también mejore el estudio y el análisis de cada deporte, y cada vez sea más específica para cada deporte, implementando nuevos recursos para mejorar los resultados. Pero, ¿son lo suficientemente eficaces los análisis que se realizan actualmente en Europa? ¿O dichos análisis estan anticuados y requieren de una actualización?

\end{abstract}

\pagebreak            % esto sirve para hacer un salto de página
\newpage              % y aquí empieza la nueva página

\tableofcontents

\clearpage



\section{Introducción}

En este trabajo estudiaremos más a fondo el Baloncesto, el segundo deporte más popular de Europa (solo superado por el futbol), y el cual yo tengo relación personal, ya que lo practico des de los 4 años. En especial nos centraremos en el Baloncesto profesional Europeo, de donde podemos obtener más datos.

Este trabajo surgió del constante pensamiento de que los análisis actuales que se hacen en este deporte en Europa son bastante pobres a nivel informativo, puesto que se basan en conceptos muy básicos. Para que nos hagamos una idea, el estadístico por preferencia es el llamado "Valoración" y que se originó en 1991 (hace 30 años) y des de entonces nunca se ha modificado.

Es por eso que, considero que actualmente los análisis que se hacen de este deporte necesitan una actualización para llegar a informar de todos aquellos datos que hoy en día si se pueden recoger gracias a los avances tecnológicos.




\end{document}
